\documentclass{article}

\usepackage{amsmath}
\usepackage{amssymb}
\usepackage{amsfonts}
\usepackage{amsthm}

\theoremstyle{definition}
\newtheorem{definition}{Definition}[section]

\newtheorem{lemma}{Lemma}

\newtheorem{corollary}{Corollary}

\title{A Family of Vertex Transitive Graphs}
\author{Edward L. Platt}

\begin{document}

\maketitle

\section{Sylvester Number System}

Let $s_n$ be the $n$th element of Sylvester's sequence \cite{sylvester1880point}, defined as:

\begin{eqnarray}
s_0 &=& 2 \\
s_{n + 1} &=& 1 + \prod_{i=0}^n s_n. \label{eq:sn}
\end{eqnarray}

A mixed-base number system can be constructed from Sylvester's sequence as follows:
\begin{definition}
A {\em Sylvester-radix} number $a$ is a sequence of digits $a_n$ such that: $a_n \in \mathbb{Z} : 0 \le a_n < s_n$.
\end{definition}

\begin{lemma}
There are $(s_n - 1)$ Sylvester-radix numbers of length $n$.
\end{lemma}

\begin{proof}
The Sylvester-radix numbers of length 1 are $(0)$ and $(1)$. $(s_1 - 1) = 2$ so the lemma holds for $n=1$.

For $n > 1$, there are $s_i$ possible values for each digit, with $0 \le i < n$. The number of valid digit combinations is thus given by:
\begin{equation*}
\prod_{i=0}^{n-1} s_i = s_n - 1 \qquad \text{(by \eqref{eq:sn})}.
\end{equation*}
\end{proof}

\begin{corollary}
The place value of index $i$ in a Sylvester-radix number is $(s_i - 1)$.
\end{corollary}

The integer value of a length-$n$ Sylvester-radix number $a$ is thus:
\begin{equation}
z(a) = \sum_{i=0}^{n-1} a_i (s_i - 1).
\end{equation}

\bibliographystyle{plain}
\bibliography{paper}

\end{document}
