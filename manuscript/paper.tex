\documentclass{article}

\usepackage[utf8]{inputenc}
\usepackage[T1]{fontenc}
\usepackage{amsmath}
\usepackage{amssymb}
\usepackage{amsfonts}
\usepackage{amsthm}
\usepackage{graphicx}
\usepackage{relsize}

\theoremstyle{definition}
\newtheorem{definition}{Definition}[section]

\newtheorem{lemma}{Lemma}
\newtheorem{corollary}{Corollary}
\newtheorem{theorem}{Theorem}

\DeclareMathOperator{\concat}{\parallel}
\DeclareMathOperator{\append}{\text{«}}

\title{A Family of Vertex Transitive Graphs}
\author{Edward L. Platt}

\begin{document}

\maketitle

\section{Sylvester Number System}

Let $s_n$ be the $n$th element of Sylvester's sequence \cite{sylvester1880point}, defined as:

\begin{eqnarray}
s_0 &=& 2 \\
s_{n + 1} &=& 1 + \prod_{i=0}^n s_n. \label{eq:sn}
\end{eqnarray}

A mixed-base number system can be constructed from Sylvester's sequence as follows:
\begin{definition}
A {\em Sylvester-radix} number $a$ is a sequence of digits such that: $a_i \in \mathbb{Z} : 0 \le a_i < s_i$.
\end{definition}

We denote the set of $n$-digit Sylvester-radix numbers as $\mathcal{S}_n$ and the set of all Sylvester-radix numbers as $\mathcal{S}$. 

\begin{lemma}
There are $(s_n - 1)$ Sylvester-radix numbers of length $n$.
\end{lemma}

\begin{proof}
The Sylvester-radix numbers of length 1 are ${\langle}0\rangle$ and ${\langle}1\rangle$. $(s_1 - 1) = 2$ so the lemma holds for $n=1$.

For $n > 1$, there are $s_i$ possible values for each digit, with $0 \le i < n$. The number of valid digit combinations is thus given by:
\begin{equation*}
\prod_{i=0}^{n-1} s_i = s_n - 1 \qquad \text{(by \eqref{eq:sn})}.
\end{equation*}
\end{proof}

\begin{corollary}
The place value of index $i$ in a Sylvester-radix number is $(s_i - 1)$.
\end{corollary}

The integer value of a length-$n$ Sylvester-radix number $a$ is thus:
\begin{equation}
z(a) = \sum_{i=0}^{n-1} a_i (s_i - 1).
\end{equation}

\section{Nested Clique}

We now define a family of graphs, which we call {\em nested cliques}. For $n > 0$, the nested clique $G_n$ is defined recursively in terms of $G_{n-1}$. We will construct $G_n$ from a union of subgraphs, each of which is isomorphic to $G_{n-1}$. Lower-order graphs are thus recursively {\em nested} within higher-order graphs. In addition to the edges internal to each subgraph, we will add one {\em external edge} to each vertex of $G_n$, such that each pair of subgraphs is connected by exactly one directed edge in each direction, making $G_n$ homomorphic to a clique.

\begin{definition} \label{def:nk}
A {nested clique sequence} is a sequence of graphs $G_n = (V_n, E_n)$ such that $G_n$ is a union of disjoint subgraphs isomorphic to $G_{n-1}$ and of a set of edges connecting each vertex in $V_n$ to another vertex belonging to a different subgraph.
\end{definition}

The above definition implies the following useful lemma.
\begin{lemma}
The number of vertices $N_n = |V_n|$ in a nested clique sequence is given by the recurrence relation:
\begin{equation}\label{eq:nn}
N_n = N_{n-1}(N_{n-1} + 1).
\end{equation}
\end{lemma}

\begin{proof}
TODO
\end{proof}

\subsection{Construction}
We now construct a particular family of graphs satisfying the requirements of Definition \ref{def:nk}.

\subsubsection{Base Case}

We choose a base case containing a single vertex, which we label $\varnothing$:
\begin{eqnarray}
G_0 &=& (V_0, E_0) \\
V_0 &=& \{ \varnothing \} \\
E_0 &=& \{\}.
\end{eqnarray}

\begin{lemma}
Whith the above base case $G_0$, the number of vertices in $G_n$ is given by:
\begin{eqnarray}
N_n &=& 1, 2, 6, 42, 1806, ...
\end{eqnarray}
\end{lemma}

\subsubsection{Vertices}

For the recursive case, it is necessary to distinguish between vertices in different nested subgraphs. Each vertex is labeled by a sequence of integers. For $G_n$ each subgraph is assigned an integer in $[0, N_n)$. For all vertices in a particular subgraph, labels are constructed by taking the corresponding label in $G_{n-1}$ and appending the subgraph's integer label. Formally:
\begin{equation}
V_n =
\bigcup_{i = 0}^{N_{n-1}}
\{ v \append i \mid v \in V_{n-1} \},
\end{equation}
where $a \append e$ denotes appending element $e$ to the end of sequence $a$. The vertices of $G_n$ are thus integer sequences of length $n$, with the digit at index $i$ in $[0, N_i-1]$. This set of labels is exactly the set of length-$n$ Sylvester-radix numbers.

\begin{theorem}
The vertices of a nested clique $G_n$ having $V_0 = \{ \varnothing \}$ are isomorphic to length-$n$ Sylester-radix numbers.
\end{theorem}

\begin{proof}
TODO
\end{proof}

\subsubsection{Edges}
It is not obvious whether any configuration of edges can satisfy Definition \ref{def:nk}. We will construct these edges below. First, we define notation that will be needed for the construction.

Let $\sigma_i$ be the cyclic permutation that increments an integer, modulo $s_i$:
\begin{equation}
\sigma_i x = (x + 1) \bmod s_i.
\end{equation}

We now define the $k$th order {\em skip operator} $S_k$. This operator on a Sylvester-radix number to permute digits 0 \ldots k. We use $\concat$ to denote the concatanetion operator for sequences.

\begin{definition}\label{def:skip}
\begin{eqnarray}
S_0 v &=& \langle \sigma_0 v_0, v_1, v_2, \ldots \rangle \\
z^\prime_i (v) &\equiv& z(S_{i-1} v_{0:i}) \\
S_i v &=&
S_{i-1} v_{0:i}
\concat \langle v_i + 1 + z^\prime_i(v) \rangle
\concat v_{i+1:n}.
\end{eqnarray}
\end{definition}


\bibliographystyle{plain}
\bibliography{paper}

\end{document}
