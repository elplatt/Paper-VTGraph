\documentclass{article}

\usepackage[utf8]{inputenc}
\usepackage[T1]{fontenc}
\usepackage{amsmath}
\usepackage{amsthm}
\usepackage{graphicx}

\newcommand{\la}{\left\langle}
\newcommand{\ra}{\right\rangle}
\newcommand{\beq}{\begin{eqnarray}}
\newcommand{\eeq}{\end{eqnarray}}

\newcommand{\append}{\oplus}

\title{DRAFT: A Family of Sparse, Vertex-Transitive Graphs}
\author{Edward L. Platt}
\date{August 2015}

\begin{document}

\maketitle

\section{Construction}

We recursively construct a family of vertex-transitive graphs
$G_n = \la V_n, E_n \ra$.
The vertices $v \in V_n$ are labeled by an $n$-sequence of integers.
We define the base case, $G_0$ as:
\beq
V_0 &=& \{\la \ra\}, \\
E_0 &=& \{\}.
\eeq
We construct subsequent vertex sets from copies of the previous set,
with each copy having a different integer appended to its vertex labels:
\beq
V_n &=& \bigcup_{k = 0}^{|V_{n-1}|}
\left\{ v | w \in V_{n-1} \land v = w \append k \right\},
\eeq
where $w \append k$ denotes appending element $k$ to the end of sequence $w$.
For convenience, we define
\beq
C_n &\equiv& |V_n|, \\
D_n &\equiv& C_{n-1} + 1.
\eeq
We note that a one-to-one mapping exists between the vertices of $G_n$ and the
integers from $0$ to $C_n - 1$:
\beq
N_n(v) = \sum_{k=0}^{n-1} C_k v_k.
\eeq

To construct the edges of $G_n$ we define an adjacency function $f_n(v,w)$
that is equal to $1$ if $v$ and $w$ are connected and equal to $0$ otherwise.
First, we define auxiliary functions:
\beq
h_n(v,w)
&=&
\begin{cases}
1 & \mbox{if } N_n(v) - N_n(w) \equiv 1 (\mbox{mod } C_n), \\
1 & \mbox{if } N_n(w) - N_n(v) \equiv 1 (\mbox{mod } C_n), \\
0 & \mbox{otherwise}.
\end{cases} \\
s_n(v,w)
&=&
\begin{cases}
1 & \mbox{if }
N_{n-1}(w_0^{n-2}) \in \{ 2, 4, \ldots, C_{n-1} - 2\} \\
&
\land \, w_{n-1} - v_{n-1} \equiv N_{n-1}(w_0^{n-2}) \, (\mbox{mod } D_n), \\
1 & \mbox{if }
N_{n-1}(v_0^{n-2}) \in \{ 2, 4, \ldots, C_{n-1} - 2\} \\
&
\land \, v_{n-1} - w_{n-1} \equiv N_{n-1}(v_0^{n-2}) \, (\mbox{mod } D_n), \\
0 & \mbox{otherwise}.
\end{cases}
\eeq
We now proceed to define $f_n$ recursively:
\beq
f_1(v,w)
&=&
\begin{cases}
0 & \mbox{if } v = w, \\
1 & \mbox{otherwise}.
\end{cases} \\
f_n(v,w)
&=&
\begin{cases}
1 & \mbox{if } f_{n-1}(v_0^{n-2}, w_0^{n-2}) = 1 \\
&
\land \, (v_{n-1} = w_{n-1}
\lor
h_n(v, w) = 1
\lor
s_n(v, w) = 1
),\\
0 & \mbox{otherwise}.
\end{cases}
\eeq

\bibliographystyle{plain}
\bibliography{vtgraph}

\end{document}
