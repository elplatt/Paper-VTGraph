\documentclass{article}

\usepackage[utf8]{inputenc}
\usepackage[T1]{fontenc}
\usepackage{amsmath}
\usepackage{amsthm}
\usepackage{graphicx}

\newcommand{\la}{\left\langle}
\newcommand{\ra}{\right\rangle}
\newcommand{\beq}{\begin{eqnarray}}
\newcommand{\eeq}{\end{eqnarray}}

\newcommand{\append}{\leftarrow{\mkern-18mu}{+}\:}

\newtheorem{lemma}{Lemma}
\newtheorem{theorem}{Theorem}

\title{DRAFT: A Family of Sparse, Vertex-Transitive Graphs}
\author{Edward L. Platt}
\date{August 2015}

\begin{document}

\maketitle

\section{Construction}

We recursively construct a family of vertex-transitive graphs
$G_n = \la V_n, E_n \ra$.
The vertices $v \in V_n$ are labeled by an $n$-sequence of integers.
We define the base case, $G_0$ as:
\beq
V_0 &=& \{\la \ra\}, \\
E_0 &=& \{\}.
\eeq
We construct subsequent vertex sets from copies of the previous set,
with each copy having a different integer appended to its vertex labels:
\beq
\label{eq:vn}
V_n &=& \bigcup_{k = 0}^{|V_{n-1}|}
\left\{ v | w \in V_{n-1} \land v = w \append k \right\},
\eeq
where $w \append k$ denotes appending element $k$ to the end of sequence $w$.
For convenience, we define
\beq
C_n &\equiv& |V_n|, \\
D_n &\equiv& C_{n-1} + 1.
\eeq
We note that a one-to-one mapping exists between the vertices of $G_n$ and the
integers from $0$ to $C_n - 1$:
\beq
N_n(v) = \sum_{k=0}^{n-1} C_k v_k.
\eeq

To construct the edges of $G_n$ we define an adjacency function $f_n(v,w)$
that is equal to $1$ if $v$ and $w$ are connected and equal to $0$ otherwise.
First, we define auxiliary functions:
\beq
d_n(v,w)
&=&
(N_n(w) - N_n(v)) \ \mbox{mod } C_n,
\\
h_n(v,w)
&=&
\begin{cases}
1 & \mbox{if }, v_{n-1} \neq w_{n-1} \land d_n(v,w) \in \{ 1, C_n - 1\}\\
0 & \mbox{otherwise}.
\end{cases}
\\
s_n(v,w)
&=&
\begin{cases}
1 & \mbox{if }
N_{n-1}(v_0^{n-2}) \in \{ 1, 3, \ldots, C_{n-1} - 3\} \\
&
\ \land \, d_{n-1}(v_0^{n-2}, w_0^{n-2}) = 1 \\
&
\ \land \, w_{n-1} - v_{n-1} \equiv N_{n-1}(v_0^{n-2}) + 1\, (\mbox{mod } D_n), \\
1 & \mbox{if }
N_{n-1}(v_0^{n-2}) \in \{ 2, 4, \ldots, C_{n-1} - 2\} \\
&
\ \land \, d_{n-1}(w_0^{n-2}, v_0^{n-2}) = 1 \\
&
\ \land \, v_{n-1} - w_{n-1} \equiv N_{n-1}(v_0^{n-2}) \, (\mbox{mod } D_n), \\
0 & \mbox{otherwise}.
\end{cases}
\eeq
We now proceed to define $f_n$ recursively:
\beq
f_1(v,w)
&=&
\begin{cases}
0 & \mbox{if } v = w, \\
1 & \mbox{otherwise}.
\end{cases} \\
\label{eq:fn}
f_n(v,w)
&=&
\begin{cases}
1 & \mbox{if } f_{n-1}(v_0^{n-2}, w_0^{n-2}) = 1 \\
&
\ \land \, (v_{n-1} = w_{n-1}
\lor
h_n(v, w) = 1
\lor
s_n(v, w) = 1
),\\
0 & \mbox{otherwise}.
\end{cases}
\eeq
The edges of $G_n$ are then:
\beq
E_n &=& \{\{ v, w \} | f_n(v,w) = 1 \}.
\eeq

\section{Properties of $G_n$}

\begin{lemma}
\label{lem:regular}
The graph $G_n$ is $n$-regular for all $n \geq 0$.
\end{lemma}
\begin{proof}
We proceed using induction on $n$.
The base case $G_0$ is trivially regular.
For the inductive case, consider vertices $v$ and $w$.
When $v_{n-1} = w_{n-1}$
Equation \ref{eq:fn} reduces to $f_{n-1}(v_0^{n-2},w_0^{n-2})$.
By assumption, $G_{n-1}$ is $(n-1)$-regular, contributing exactly
$n-1$ edges to the degree of $v$.

In the case that $v_{n-1} \neq w_{n-1}$ and
$N_{n-1}(v_0^{n-2}) \in \{ 1, 2, \ldots, C_{n-1} - 3\}$,
$h_n(v) = 0$ and $s_n(v) = 1$ if
\beq
w &=& v_0^{n-2} \append (v_{n-1} + N_{n-1}(v_0^{n-2}) + 1) \ \mbox{mod } D_n,
\eeq
adding another edge to $v$ for a total degree of $n$.
The case for $v_{n-1} \neq w_{n-1}$ and
$N_{n-1}(v_0^{n-2}) \in \{ 2, 4, \ldots, C_{n-1} -2 \}$ is analogous.

In the remaining case, $v_{n-1} \neq w_{n-1}$ and
$N_{n-1}(v_0^{n-2}) \not\in \{ 1, 3, \ldots, C_{n-1} - 2 \}$.
For all such $v$, $s_n(v,w) = 0$, and
there exists exactly one $w$ such that $h_n(v,w) = 1$, bringing the total
degree of $v$ to $n$.
As we have not placed any constraints on $v$, all vertices of
$G_n$ have degree $n$.
\end{proof}

\begin{lemma}
The number of vertices and edges of the graph $G_n$ are given by the recurrence
relations:
\beq
C_0 = |V_0| &=& 1, \\
\label{eq:cnrec}C_{n} = |V_{n}| &=& C_{n-1} (C_{n-1} + 1), \\
|E_n| &=& \frac{n}{2} C_n.
\eeq
\end{lemma}
\begin{proof}
The vertex set of $G_n$ (Equation \ref{eq:vn}) is a union of
$C_{n-1} + 1$ sets.
The vertex labels within each set all end with the same element,
and this element is unique to each set.
The sets are thus disjoint.
Each set contains $C_{n-1}$ elements, giving $C_{n-1}(C_{n-1} + 1)$
elements.
By Lemma \ref{lem:regular}, $G_n$ is $n$-regular and has
$\frac{n}{2}|V_n| = \frac{n}{2}C_n$ edges.
\end{proof}

\begin{theorem}
The graphs $G_n$ are vertex-transitive for all $n \geq 0$.
\end{theorem}
\begin{proof}
Define $\phi$, show preserves edges.
\end{proof}

\bibliographystyle{plain}
\bibliography{vtgraph}

\end{document}
